\chapter{Preliminaries}
\pagenumbering{arabic}
\section{Historical Notes}

\begin{quote}
    The axiom of choice is probably the most interesting and, in spite of its late appearance, the most discussed axiom of mathematics,
    second only to Euclid's axiom of parallels which was introduced more than two thousand years ago. \autocite{fraenkel1973}
\end{quote}
The axiom of choice was first explicitly mentioned in a paper on existence of solutions to differential equations by Giuseppe Peano in 1890,
where he alluded to its negation.

Circa 1901, Georg Cantor and Felix Bernstein were trying to construct a bijection between the continuum and the set of all countable order types (Cantor-Bernstein theorem), when they faced difficulty. Beppe Levi suggested to resolve the difficulty by introducing choice which he formulated in a general form.

The first explicit formulation of choice was by Zermelo in 1904, where he used it to prove the well-ordering theorem.
However, back then the set-theoretical notions of ordered pair and function were not yet defined, so he formulated it using a looser notion of ``functional correspondence''.

In 1906, Bertrand Russell formulated a modern form of choice, which states that \(\prod t\) is nonempty if \(t\) is a disjointed set and it does not contain the empty set.

% https://plato.stanford.edu/entries/axiom-choice/

Zermelo's 1904 introduction of the axiom was very controversial for its day, mathematicians of that time objected to its non-constructive nature.
While the axiom asserts the possibility of making infinitely many, or even uncountably many ``choices'', it gives no procedure on how those choices could be made.

As the debate raged on, it became apparent that a number of significant mathematical results hinges on the axiom.
David Hilbert notably regarded choice as an ``indispensible'' principle of mathematics.

Another source of objections is that the axiom of choice is a source of unintuitive and unpleasant antimonies and paradoxes,
theorems which are not in line with our ``common sense'' intuition.
The most spectacular of these was Banach and Tarski's \emph{paradoxical decomposition of the sphere},
where one cuts a ball into finitely many pieces, rearranges them, and gets two balls each of same size as the original one.
Another consequence is the existence of Vitali set, a set of real numbers which is not Lebesgue measurable.

It was not until 1930s that Kurt Gödel proved the relative consistency of Choice with respect to the other axioms of set theory.
This puts the soundness problem to rest by showing that adding Choice to \ZF does not make the theory any less consistent.

For a detailed history of the Axiom of Choice, see \autocite{moore1982} and \autocite{fraenkel1973} II \S 4.


\section{ZFC Axioms}

For completeness, we shall list the axioms of \ZFC (shorthand for Zermelo-Fraenkel Set Theory with Axiom of Choice).
\ZFC is a first order theory in the language with a single binary symbol \(\in\).
The theory attempts to encapsulate what holds true in the structure \((\V,\in)\) where \(\V\) is the ``universe of all sets''.

\begin{axiom}[Extensionality]
    \[ \forall x \forall y \left( \forall z\left(z\in x\leftrightarrow z\in y\right) \rightarrow x = y \right). \]
    Extensionality tells us that two sets are equal if and only if they have the same members.
\end{axiom}

\begin{axiom}[Foundation]
    \[ \forall x \sq{ \exists y \paren{y\in x} \to \exists y \paren{y\in x\land \lnot\exists z\paren{z\in x\land x\in y}} } .\]
    Foundation says that every nonempty \(x\) has a \(\in\)-minimal member.
\end{axiom}

\begin{axiom}[Comprehension Scheme]
    For each formula \(\phi\) with free variables \(x,z,w_1,\dots,w_n\),
    \[ \forall z \forall w_1,\dots,w_n\ \exists y\ \forall x\paren{x\in y\leftrightarrow x\in z\land \phi}. \]
    As an axiom scheme, for well-formed logical formula \(\phi\), there is an instance of the axiom.
    Intuitively it states that for every set \(z\) and property \(\phi(x, z, \overline{w})\), there is a subset \(y\) of \(z\) whose members are precisely those \(x\in z\) for which \(\phi(x)\) holds. We denote \(y\) as \[ \set{x\in z: \phi} .\]
    Comprehension allows us to define the empty set \(0 = \set{x: x\ne x}\).
\end{axiom}

\begin{axiom}[Pairing]
    \[\forall x\ \forall y\ \exists z\paren{x\in z\land y\in z}.\]
    Together with axiom of comprehension, it follows that \(\set{x,y}\) exists.
\end{axiom}

\begin{axiom}[Union]
    \[\forall\mathcal{F}\ \exists A\ \forall Y\ \exists x\paren{x\in Y\land Y\in\mathcal{F}\to x\in A}. \]
    This axiom says that for every set \(\mathcal{F}\), there is a set \(A\) which contains the union of all the sets in \(\mathcal{F}\).
    Together with axiom of comprehension, it implies that every set has a union.
\end{axiom}

\begin{axiom}[Replacement Scheme]
    For each formula \(\phi\) with free variables \(x,y,A,w_1,\dots,w_n\), \[ \forall A\ \forall w_1,\dots,w_n \sq{\forall x\in A\ \exists! y\ \phi \to \exists Y\ \forall x\in A\ \exists y\in Y\ \phi} .\]
    Replacement says that if there is a formula \(\phi(x,y,A)\) which defines a ``function'' \(x\mapsto y\) on the set \(A\),
    then there exists a set \(B\) which contains the image of this ``function'' on \(A\).
\end{axiom}

\begin{axiom}[Infinity]
    \[\exists x\paren{0\in x\land \forall y\in x\paren{y\cup\set{y} \in x}}. \]
    Infinity asserts the existence of \(\omega\) the set of all natural numbers.
\end{axiom}

\begin{axiom}[Power Set]
    \[\forall x\ \exists y\ \forall z\paren{z\subseteq x\to z\in y}.\]
\end{axiom}

\begin{axiom}[Choice]
    \[ \forall A\ \exists R \paren{R \text{ well-orders } A}. \]
\end{axiom}

Nevertheless, there are mathematicians who either do not ``believe in'' the axiom of choice or who are interested in the logical repercussions of disallowing this axiom in their set theory.

Throughout this thesis, \ZFC will denote the theory of all the above axioms, \ZF will denote \ZFC without Choice,
and \ZFCminus and \ZFminus denotes \ZFC and \ZF respectively, but without foundation.

\section{todo}

\begin{itemize}
    \item consistency proofs
    \item relativisation
    \item absoluteness
\end{itemize}

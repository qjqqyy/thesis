\chapter{Basic Fraenkel Model}
\label{chapter:fraenkelmostowski}

The basic Fraenkel model is not a model of \ZF where choice fails, but rather a model of \textsf{ZFA}, that is \ZF with atoms, or urelements, where choice fails.
The methods used here were introduced by Fraenkel in 1922 and further developed by Mostowski in 1938,
which is over 30 years before the development of forcing.
This model offers a good primer on how choice can fail due to its relative technical simplicity.

\section{ZF with atoms}

This section will introduce the theory \textsf{ZFA}, that is \ZF with atoms and
show that in \ZFC, one can produce a model of \textsf{ZFA} with infinitely many atoms.

\newcommand*{\F}{\mathbf{F}}
\renewcommand*{\Im}{\operatorname{Im}}
Let \(\mathbf{F}:\V\to\V\) be any permutation of the universe (one-one and onto class function),
it turns out that we can use \(\mathbf{F}\) to redefine the membership relation and obtain a model without foundation.
We will make this precise as follows,
\begin{lemma} % Kunen chapter IV exercise 18
    \label{KunenC4E18}
    Working in \ZFminus, let \(\mathbf{F}:\V\to\V\) be one-one and onto.
    Define \(\E\subseteq \V\times\V\) by \(x\ \E\ y\) iff \(x\in \mathbf{F}(y)\),
    then \(\V,\E\) is a model of \ZFminus
    and \({\normalfont\textsf{Choice}}\) implies \({\normalfont\textsf{Choice}}^{\V,\mathbf{E}}\).
\end{lemma}
\begin{proof}
    We relativise all \(\ZFCminus\) axioms to \(\V,\E\) by replacing all occurrences of \(x\in y\) with \(x\in \F(y)\)
    and show that they hold.
    \begin{itemize}
        \item \(\textsf{Extensionality}^{\V,\mathbf{E}}\) is
            \[ \forall x~\forall y \paren{
                    \forall z \paren{z\in \F(x) \leftrightarrow z\in \F(y)} \rightarrow x = y
            },\]
            let \(x, y\in \V\), using extensionality (in \(\V\)), \(\F(x) = \F(y)\), and as \(\F\) is 1-1 we know \(x = y\).

        \item Let \(\phi\) be a formula with free variables \(x,z,\overline{w}\), its instance of \(\textsf{Comprehension}^{\V,\mathbf{E}}\) is
            \[ \forall z~\forall \overline{w}~\exists y~\forall x \paren{
                    x \in \F(y) \leftrightarrow x\in \F(z) \land \phi
            },\]
            let \(z,\overline{w}\in \V\), comprehension (in \(\V\)) gives us existence of \(\F(y)\) and use fact that \(\F\) is a permutation to get existence of \(y\).

        \item For \(\textsf{Union}^{\V,\mathbf{E}}\) which is
            \[
                \forall \mathscr{F}~\exists A~\forall Y~\forall x\paren{x\in \F(Y)\land Y\in\F(\mathscr{F}) \rightarrow x\in\F(A)},
            \]
            let \(\mathscr{F}\in\V\),
            apply union (in \(\V\)) to \(\Im_{\F}(\F(\mathscr{F}))\) (appealing to replacement in \(\V\)),
            then there exists \(B\) satisfying
            \begin{equation}\label{c2:eqn:union}
                \forall Z~\forall w\paren{w\in Z\land Z\in \Im_{\F}(\F(\mathscr{F})) \rightarrow w\in B}.
            \end{equation}
            Let \(A = \F^{-1}(B)\) and let \(Y,x\in\V\), assume \(x\in \F(Y)\) and \(Y\in\F(\mathscr{F})\),
            let \(Z = \F(Y)\) and \(w = x\), apply \Cref{c2:eqn:union} to conclude that \(x\in \F(A) = B\).

        \item Let \(\phi\) be a formula with free variables \(x,y,A,\overline{w}\), its instance of \(\textsf{Replacement}^{\V,\E}\) is
            \[ \forall A~\forall \overline{w} \sq{
                    \forall x\in\F(A) ~ \exists!y\ \phi \rightarrow \exists Y~ \forall x\in \F(A)~ \exists y\in \F(Y)~ \phi
                },
            \]
            let \(A,\overline{w}\in \V\) and assume that \(\phi\) defines a function on \(\F(A)\).
            We need to find \(Y\) such that \(\Im_{\F}(A) \subseteq \F(Y)\), and \(\F^{-1}(\Im_{\F}(A))\) satisfies the condition.

        \item For \(\textsf{Choice}^{\V,\E}\) we can verify that it is in fact
            \[ \forall A~\forall R\paren{\F(R) \text{ well-orders } \F(A)} \]
            which follows from \textsf{Choice} in \(\V\).
    \end{itemize}
    The other axioms can be shown to hold in \(\V,\E\) with similar methods.
\end{proof}

\begin{example}\label{babyAtomExample}
    Assuming consistency of \ZFC, the theory \(\ZFCminus + \exists x~\paren{x = \set{x}}\) is consistent.
\end{example}
\begin{proof}
    Define \(\F\) such that \(\F(0) = \set{0}\) and \(\F(\set{0}) = 0\) and \(\F\) is identity everywhere else,
    so \(\F\) is a permutation of \(\V\) that interchanges \(0\) and \(1\).
    Apply \Cref{KunenC4E18},
    since \(0 \in \set{0} = \F(0)\), \((0\in 0)^{\V,\E}\),
    so in \(\V,\E\), it follows that \((\set{0} \subseteq 0)^{\V,\E}\).
    As \(0 \subseteq \set{0}\) it follows that \((0 = \set{0})^{\V,\E}\).
\end{proof}

\newcommand{\WF}{\mathbf{WF}}
\newcommand{\trcl}{\operatorname{trcl}}
The axiom of foundation tells us that no set can be a member of itself.
What we have just done is produce a model of \ZFCminus with a single indecomposable element -- \(0\), which essentially behaves like an atom.

\begin{definition}
    Given a set \(A\), define \(R(\alpha,A)\) by
    \begin{itemize}
        \item \(R(0,A) = A \cup \operatorname{trcl}(A)\),
        \item \(R(\alpha+1, A) = \mathcal{P}(R(\alpha, A))\), and
        \item \(R(\alpha, A) = \bigcup_{\xi<\alpha} R(\xi, A)\) when \(\alpha\) is a limit ordinal.
    \end{itemize}
    Let \(\WF(A) = \bigcup_{\alpha\in\mathbf{ORD}} R(\alpha, A)\).
\end{definition}

\begin{lemma} \label{WFfromA}
    Working in \ZFminus, \(\WF(A)\) is a transitive model of \ZFminus and {\normalfont\textsf{Choice}} implies \({\normalfont\textsf{Choice}}^{\WF(A)}\) is consistent.
\end{lemma}
\begin{proof}
    \(\WF(A)\) being transitive is obvious and all transitive models satisfy \Axiom{Extensionality} by \Cref{lemma:transitive_extensional}.

    \Axiom{Comprehension} follows from \(\WF(A)\) being closed under power set.

    \Axiom{Power set} follows from \Cref{lemma:power_set}.

    \Axiom{Pairing} and \Axiom{Union} looks easy.

    \Axiom{Replacement} -- For any definable class function \(\F\) and any set \(X\in\WF(A)\), take minimum \(\alpha\) such that the \(\mathbf{F}\)-image of \(X\) is contained in \(R(\alpha, A)\).

    \Axiom{Infinity} looks obvious.

    Assume \Axiom{Choice} and \(\Axiom{Choice}^{\WF(A)}\) holds since \Axiom{Choice} can be expressed as a \(\Pi_1\) statement which relativises down.
\end{proof}

Now we can produce a model of \textsf{ZFA} with infinitely many atoms, as hinted at the start of the section.
\begin{theorem}
    Assuming consistency of \ZFminus, the theory of \(\ZFminus + (\V=\WF(A))\) where \(A\) is an infinite set satisfying \(\forall x\in A\paren{x = \set{x}}\).
\end{theorem}
\begin{proof}
    Define \(\F:\V\to\V\) such that for each \(n\in\omega\),
    \[ \F(n) = \set{n} \text{ and } \F(\set{n}) = n \]
    and \(\F\) is identity everywhere else.
    Apply \Cref{KunenC4E18}, by an argument similar to \Cref{babyAtomExample} we see that for each \(n\in\omega\), \((n = \set{n})^{\V,\E}\).

    Work in \(\V,\E\) which is a model of \ZFminus where \(A = \omega\) satisfies \(\forall x\in A\paren{x = \set{x}}\).
    Applying \Cref{WFfromA} we still have a model of \ZFminus and \(\forall x\in A\paren{x = \set{x}}\) still holds by \(\Delta_0\)-absoluteness.
\end{proof}

\section{Fraenkel-Mostowski Theorem}

\newcommand{\Con}{\operatorname{Con}}
In this section, we work in a universe \(\V = \WF(A)\) that satisfies \ZFminus, where \(A\) is the infinite set of all atoms.
We aim to show the result of Fraenkel-Mostowski, that being
\(\Con(\ZFminus) \to \Con(\ZFminus + \lnot\textsf{Choice})\).

Consider \(G\) the group of permutations on \(A\),
as \(\V\) is built by iterating the powerset operation starting from \(A\),
for any \(\pi\in G\) we can naturally extend \(\pi\) to be a permutation \(\V\to\V\).
\begin{definition}
    Let \(\pi \in G\), abusing notation, we extend \(\pi\) by
    \begin{itemize}
        \item \(\pi(a) = a\) for atomic \(a\in A\),
        \item \(\pi(y) = \set{\pi(z): z\in y}\) otherwise.
    \end{itemize}
\end{definition}
\begin{remark}
    Formally, this definition is achieved by recursion on the \(A\)-rank of \(x\in\V\), which is \(\min_{\alpha} \set{x\in R(\alpha,A)}\).
\end{remark}
\begin{observation}
    The following properties hold for \(\pi:\V\to\V\),
    \begin{enumerate}
        \item \(x\in y\leftrightarrow \pi x \in \pi y\).
        \item \(\pi(\set{x,y}) = \set{\pi x, \pi y}\).
        \item \(\pi(\paren{x,y}) = \paren{\pi x, \pi y}\).
    \end{enumerate}
\end{observation}

\begin{definition}
    For each \(B\subseteq A\) we denote its stabilizer
    \[ G_B = \set{\pi\in G: \forall x\in B\paren{\pi x = x}}.\]
    Define the class of all \emph{symmetric objects} as
    \[ \mathbf{S} = \set{y:\exists B\subseteq A\paren{\abs{B}<\omega \land \forall \pi\in G_B\paren{\pi y = y}}}. \]
    Then \(\mathbf{HS} = \set{y\in \mathbf{S}:\trcl(y)\subseteq \mathbf{S}}\) will denote all \emph{hereditarily symmetric} objects.
\end{definition}
\begin{remark}
    In this case \(\mathbf{HS}\) is what we call the \textbf{Basic Fraenkel model}.
\end{remark}

\begin{theorem} \label{theorem:f-k-model-zfminus}
    \(\mathbf{HS}\) is a transitive model of \ZFminus.
\end{theorem}
\begin{proof}
    Transitivity is obvious.

    To show that \(\mathbf{HS}\) is closed under Gödel operations,
    we illustrate by showing that if \(x,y\) are symmetric then so is \(\set{x,y}\).
    Let \(B, B'\) be finite sets witnessing \(x\in \mathbf{S}\) and \(y\in\mathbf{S}\) respectively,
    then \(B\cap B'\) is still finite and witnesses \(\set{x,y} \in \mathbf{S}\).
    The 7 other Gödel operations are similar.

    To show that \(\mathbf{HS}\) is almost universal,
    it suffices to show that for each \(\alpha\),
    \(R(\alpha,A) \cap \mathbf{HS}\) is symmetric.
    Note that for all \(x\), if \(B\) witnesses \(x\in \mathbf{S}\),
    then \(\pi''B\) witnesses \(\pi x\in \mathbf{S}\) for any permutation \(\pi\).

    Since the \(A\) rank of \(\pi x\) is the same as that of \(x\),
    we have \(\pi\paren{R(\alpha,A) \cap \mathbf{HS}} = R(\alpha,A) \cap \mathbf{HS}\)
    for all \(\alpha\), for all \(\pi\).

    Since \(\mathbf{HS}\) is transitive, almost universal and closed under Gödel operations, by \Cref{theorem:jech_zf_model} \(\mathbf{HS}\) is a model of \ZFminus.
\end{proof}

\begin{theorem}[Fraenkel-Mostowski]
    \(A\in\mathbf{HS}\) but \(A\) cannot be ordered by \(\mathbf{HS}\), so in particular
    \(\mathbf{HS}\) is a model of \(\ZFminus + \lnot{\normalfont\textsf{Choice}}\).
\end{theorem}
\begin{proof}
    Since \(\pi(A) = A\) for all \(\pi\in G\), \(A\) is symmetric.
    Suppose the result fails and work in \(\mathbf{HS}\), let \(R\) be an ordering on \(A\).
    As \(R\) is symmetric, let \(B\subseteq A\) be finite such that each \(\pi\in G_B\) fixes \(R\).
    We can compute that
    \[ \pi(R) = \set{\paren{\pi x,\pi y}: \paren{x,y}\in R} = R \]
    so \(x\,R\,y\) iff \(\pi x\,R\,\pi y\).
    However, this fails to hold in general as \(\pi\) is only known to fix \(B\), a finite set, this is the contradiction.
\end{proof}


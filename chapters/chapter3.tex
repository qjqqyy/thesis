\chapter{Cohen Models}

Paul Cohen invented forcing to deal with CH in 1963 and used it to prove the independence of Choice from \ZF and the independence of CH from \ZFC.
TODO: more history
% cohen essay here: http://www.logic.univie.ac.at/~ykhomski/ST2013/The%20Discovery%20of%20Forcing.pdf

\section{Forcing}

TODO: here or chapter 1?



\section{Symmetric Extensions}

\renewcommand*{\G}{\mathscr{G}}
\newcommand*{\dom}{\operatorname{dom}}
\renewcommand*{\P}{{\mathbb{P}}}
Let \(M\) be a transitive model of ZFC, consider a set \((\mathbb{P}, \leq, 1_{\mathbb{P}}) \in M\) of forcing conditions,
let \(\G\) be a group of automorphisms \(\mathbb{P} \to \mathbb{P}\).
Each \(\pi\in \G\) can be naturally extended into an automorphism \(\pi: M^{\mathbb{P}} \to M^{\mathbb{P}}\) by
\[ \pi(x,p) = \set{(\pi_* y , \pi p ): y\in x}. \]

\begin{definition}
    Let \(\mathscr{F}\) be a nonempty set of subgroups of \(\G\).
    \(\mathscr{F}\) is a \textbf{normal filter} iff it is a filter and is also closed under congujation, that is
    if \(\pi\in \G\) and \(H\in\mathscr{F}\) then \(\pi H \pi^{-1} \in\mathscr{F}\).
\end{definition}
Given a normal filter \(\mathscr{F}\) we can use it to define the notion of *symmetric*.
\begin{definition}
    A \(\mathbb{P}\)-name \(x\in M^{\mathbb{P}}\) is \textbf{symmetric} if its stabilizer \(\G_x = \set{\pi: \pi x = x}\) is in \(\mathscr{F}\).

    A name \(x\in M^{\mathbb{P}}\) is \textbf{hereditarily symmetric} if \(x\) is symmetric and every member in \(\dom(x)\) is hereditarily symmetric.
    Denote set of hereditarily symmetric names as \(\mathbf{HS}\).
\end{definition}

Given any generic filter \(G\), consider the subset of \(M[G]\) formed by evaluating all the names in \(\mathbf{HS}\),
\[ N = \set{ x[G]: x\in M^{\mathbb{P}} }. \]
Observe that we have the embeddings \(M \subseteq N \subseteq M[G]\).

\begin{theorem}
    \(N\) is a model of \ZF.
\end{theorem}
\begin{proof}
    TODO: annoying, later
\end{proof}

\section{Basic Cohen Model}

The Basic Cohen Model was the first model produced in which choice fails.
Together with Gödel's earlier results on \(\mathbf{L}\), this completes the proof of the independence of choice.

We force with \(\mathbb{P} = \operatorname{Fn}(\omega\times\omega, 2)\)
the set of finite partial functions \(\omega\times\omega \rightharpoonup 2\),
ordered by reverse containment \(\leq = \supseteq\),
with \(\mathbb{1}_\P = 0\) the empty function.

The idea is to add \(\omega\) many reals to our symmetric extension satisfying
\begin{equation} \label{basicCohen:property_x_n}
    x_n = \set{m\in\omega: \left(\exists p\in G\right) p(n,m) = 1}.
\end{equation}
\(N\) should contain \( A = \set{x_n:n\in \omega } \) but not know how to well-order it.

Come up with \(\P\)-names for objects we wish to add
\[ \mathring{x}_n = \set{(\check{m},p): m\in\omega, p\in\P,p(n,m) = 1} \]
\[ \mathring{A} = \set{(\mathring{x}_n,1_\P): n\in\omega} \]


Any permutation \(\pi: \omega \to \omega\) induces an automorphism \(\pi: \mathbb{P}\to\mathbb{P}\) as
\[ \pi = \set{((\pi n, m), y): ((n,m),y) \in p} \]
so we are extending a permutation on \(\omega\) into one on \(\operatorname{Fn}(\omega\times\omega, 2)\)
by applying it to the first coordinate of the domain.

Let \(\G\) be all such induced permutations on \(\mathbb{P}\), then for any finite \(B\subseteq \omega\),
we look at permutations induced by ones in its stabilizer
\[ \G_B = \set{ \pi\in \G: \forall n\in B\left( \pi n = n \right) }. \]

Use stabilizers of all finite subsets to generate a normal filter
\[ \mathscr{F} = \set{H: \exists \text{ finite } B\subseteq \omega \left(\G_B \subseteq H\right) }. \]
TODO: check it's a normal filter

Let \(\G\) and \(\mathscr{F}\) determine \(\mathbf{HS}\) and if we let \(G\) be a generic filter, we yield a symmetric model \(N\).

\begin{lemma}
    \(\pi \mathring{x}_n = \mathring{x}_{\pi n} \).
\end{lemma}
\begin{proof}
    TODO: computation
\end{proof}
\begin{corollary}
    \(\mathring{x}_n\) is symmetric as its stabilizer contains \(G_\set{n}\).
\end{corollary}

\begin{lemma}
    \(N\models x_i\ne x_j\) whenever \(i\ne j\)
\end{lemma}
\begin{proof}
    Suppose for contradiction that \(N\models x_i\ne x_j\),
    then let it be forced by \(p\in \P\) such that \(p\Vdash \mathring{x}_i = \mathring{x}_j\).
    As \(p\) is finite let \(m\in\omega\) such that \(p\) is not defined at both \((i,m)\) and \((j,m)\).
    We can then find an extension \(q\leq p\) with \(q(i,m) = 1\) and \(q(j,m) = 0\),
    applying \Cref{basicCohen:property_x_n}
    we see that \(q\) forces \(m\in x_i \land m\notin x_j\).
\end{proof}

\begin{lemma}
    There does not exist a bijection \(f:\omega\to A\).
\end{lemma}
\begin{proof}
Suppose there is, let \(p_0 \in G\) such that \[p_0\Vdash \mathring{f} \text{ is a bijection } \check{\omega}\to\mathring{A}. \]
As \(\mathring{f}\) is symmetric, let \(B\subseteq \omega\) be finite such that \(\G_B\) is contained in stabilizer of \(\mathring{f}\).
Let \(n\notin B, i\in\omega, p\leq p_0\) such that \[p\Vdash \mathring{f}(\check{i}) = \mathring{x}_n.\]
Now we can find \(\pi\in\G\) satisfying
\begin{enumerate}[i.]
    \item \(\pi p\) compatible with \(p\),
    \item \(\pi \in \G_B\),
    \item \(\pi n\ne n\).
\end{enumerate}
Then \(\pi p \Vdash (\pi \mathring{f})(\pi \check{i}) = \pi \mathring{x}_{n}\), so
\[p \cup \pi p \Vdash \mathring{f}(\check{i}) = \mathring{x}_n \land \mathring{f}(\check{i}) = \mathring{x}_{\pi n}. \]
This contradiction completes the proof.
\end{proof}

\begin{theorem}
    The axiom of choice fails in \(N\) and is therefore independent of \ZF.
\end{theorem}




\section{Second Cohen Model}

In the second Cohen Model, the axiom of choice fails on a set which \(N\) knows is countable.


We force with \(\mathbb{P} = \operatorname{Fn}((\omega\times 2 \times\omega)\times\omega, 2)\)
the set of finite partial functions \((\omega\times 2 \times\omega)\times\omega \rightharpoonup 2\),
ordered by reverse containment \(\leq = \supseteq\),
with \(\mathbb{1}_\P = 0\) the empty function.

Define the notions
\begin{itemize}
    \item \(x_{n\epsilon i} = \set{j\in\omega: \left(\exists p\in G\right) \left(p(n\epsilon i, j) = 1\right) }\)
    \item \(X_{n\epsilon} = \set{x_{n\epsilon i}: i\in\omega}\)
    \item \(P_n = \set{X_{n0}, X_{n1}}\)
    \item \(A = \set{P_n:n\in\omega}\)
\end{itemize}

TODO: P-names for these

Again we can extend permutations on \((\omega\times2\times\omega)\) to permutations on \(\P\) naturally
\[\pi p = \set{ ((\pi(n\epsilon i), j), y): (((n\epsilon i),j), y) \in p }.\]
Restrict our attention to permutations \(\pi\) satisfying
\begin{enumerate}[i.]
    \item \(n' = n\),
    \item for each \(n\) either \(\forall i\paren{\epsilon' = \epsilon}\) or \(\forall i\paren{\epsilon' \ne \epsilon}\)
\end{enumerate}
where \((n'\epsilon' i') = \pi(n\epsilon i)\).
Consider the automorphism group \(\G\) of all such permutations extended to \(\P\),
for any finite \(B\subset (\omega\times2\times\omega)\),
\[ \G_B = \set{ \pi\in \G: \forall n\in B\left( \pi n = n \right) }. \]
Again we can generate a normal filter, and proceed to get symmetric model.


\begin{lemma}
    \(x_{n\epsilon i}, X_{n\epsilon}, P_n, A\) all in \(N\) (their names are symmetric).
\end{lemma}
\begin{proof}
    TODO: soon...
\end{proof}

\begin{lemma}
    \(N\models A\text{ is countable}\).
\end{lemma}
\begin{proof}
  Define \(\mathring{g} = \set{((\check{n}, \mathring{P}_n), 1_\P): n\in\omega}\), check that \(\mathring{g}\) is \(\mathbf{HS}\).
  Evaluating \(\mathring{g}\) will enumerate \(\left\langle P_n: n\in\omega\right\rangle\).

  TODO: need to elaborate on this ordered pair, which is actually a name.
\end{proof}

\begin{theorem}
    \(N\) does not contain a choice function on \(A\).
\end{theorem}
\begin{proof}
Suppose \(f\) is a choice function on \(A\), let \(\mathring{f}\in\mathbf{HS}\) be its symmetric name and \(p_0\in G\) force
\[p_0\Vdash \mathring{f}\text{ is a function with domain } \mathring{A} \text{ and } \mathring{f}(\mathring{P}_n) \in \mathring{P}_n\text{ for all }n .\]
Let \(B\subseteq \omega\times2\times\omega\) such that stabiliser of \(\mathring{f}\) contains \(\G_B\),
let \(n\in\omega, p\leq p_0, \epsilon_0\) such that (wlog suppose \(\epsilon_0 = 0\))
\[ p\Vdash \mathring{f}(\mathring{P}_n) = \mathring{X}_{n0}. \]
We can find \(\pi\in\G\) satisfying
\begin{enumerate}[i.]
    \item \(\pi p\) compatible with \(p\)
    \item \(\pi \in \G_B\)
    \item \(\pi(\mathring{X}_{n0}) = \pi(\mathring{X}_{n1})\)
\end{enumerate}
Observe that \(i\)-coordinate is ``free'',
we are free to shuffle the sequence \(\langle x_{n\epsilon i}:i\in\omega\rangle\).
Let \(\pi\) be extended from
\begin{align*}
    \pi(n,0,i) &=
    \begin{cases}
        (n,1,i+k) &\text{ when } i < k \\
        (n,1,i-k) &\text{ when } k \leq i < 2k \\
        (n,1,i) &\text{ otherwise}
    \end{cases} \\
    \pi(n,1,i) &= \text{similar to above} \\
    \pi(m,\epsilon,i) &= (m,\epsilon,i) \quad\text{ whenever }m \ne n
\end{align*}

With \(\pi\) obtained, use similar forcing argument as Cohen basic model to get
\[ N\models f\text{ is not a function}.\]
\end{proof}
